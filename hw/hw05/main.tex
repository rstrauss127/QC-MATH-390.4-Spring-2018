\documentclass[12pt]{article}
\usepackage{bbm}
\include{preamble}

\newtoggle{professormode}
%\toggletrue{professormode} %STUDENTS: DELETE or COMMENT this line



\title{MATH 390.4 / 650.2 Spring 2018 Homework \#5t}

\author{Professor Adam Kapelner} %STUDENTS: write your name here

\iftoggle{professormode}{
\date{Due 11:59PM Friday, May 18, 2018 under the door of KY604 \\ \vspace{0.5cm} \small (this document last updated \today ~at \currenttime)}
}

\renewcommand{\abstractname}{Instructions and Philosophy}

\begin{document}
\maketitle

\iftoggle{professormode}{
\begin{abstract}
The path to success in this class is to do many problems. Unlike other courses, exclusively doing reading(s) will not help. Coming to lecture is akin to watching workout videos; thinking about and solving problems on your own is the actual ``working out.''  Feel free to \qu{work out} with others; \textbf{I want you to work on this in groups.}

Reading is still \textit{required}. For this homework set, read about all the concepts introduced in class online e.g. probabilistic classification, the logistic link function, performance characteristics of binary classification, asymmetric cost / reward classifiers, bias-variance decomposition, bagged models, RandomForests$^\circledR$, correlation, causation, lurking variables and graphical depictions of causal models. It is your responsibility to supplement the notes with your own readings. Also, read the introduction in Finlay.

The problems below are color coded: \ingreen{green} problems are considered \textit{easy} and marked \qu{[easy]}; \inorange{yellow} problems are considered \textit{intermediate} and marked \qu{[harder]}, \inred{red} problems are considered \textit{difficult} and marked \qu{[difficult]} and \inpurple{purple} problems are extra credit. The \textit{easy} problems are intended to be ``giveaways'' if you went to class. Do as much as you can of the others; I expect you to at least attempt the \textit{difficult} problems. 

This homework is worth 100 points but the point distribution will not be determined until after the due date. See syllabus for the policy on late homework.

Up to 10 points are given as a bonus if the homework is typed using \LaTeX. Links to instaling \LaTeX~and program for compiling \LaTeX~is found on the syllabus. You are encouraged to use \url{overleaf.com}. If you are handing in homework this way, read the comments in the code; there are two lines to comment out and you should replace my name with yours and write your section. The easiest way to use overleaf is to copy the raw text from hwxx.tex \emph{and} preamble.tex into two new overleaf tex files with the same name. If you are asked to make drawings, you can take a picture of your handwritten drawing and insert them as figures or leave space using the \qu{$\backslash$vspace} command and draw them in after printing or attach them stapled.

The document is available with spaces for you to write your answers. If not using \LaTeX, print this document and write in your answers. I do not accept homeworks which are \textit{not} on this printout. Keep this first page printed for your records.

\end{abstract}

\thispagestyle{empty}
\vspace{1cm}
NAME: \line(1,0){380}
\clearpage
}


\problem{These are questions about the Finlay's introduction to his book.}

\begin{enumerate}

\easysubproblem{Finlay introduces predictive analytics by using the case study of what supervised learning problem? Explain.}\spc{4}

Finlay introduces predictive analytics with a credit scoring model. A persons credit score is calculated from a scorecard of features. Each feature has a range levels. For example, in the employment status feature, you add 28 points to your score if you have a full time job. However, if you're unemployed, you must subtract 42 points from your score. A high credit score indicates that a person is likely to pay back their loan.

\hardsubproblem{What does a credit score of 700 mean? Use figure 1.2 on page 5 when answering this question.}\spc{3}

A credit score of 700 corresponds to 1024:1 odds. This implies that out of 1025 people who take out a loan with a credit score of 700, 1024 will repay their loans. In other words, a person with a credit score of 700 has approximately a $0.1\%$ of defaulting. 

\hardsubproblem{How much more likely is someone to default if that have 9 or more credit cards than someone with 4-8 credit cards?}\spc{3}

Everyone starts with a score of 670 according to the table. Someone if 9 or more credit cards subtracts 18, bringing their score down to 652$\approx$224:1 odds. Someone with 4-8 cards adds 0 to their score, remaining at 670$\approx$320:1 odds. 
So a person with +9 credits is approxiemetly 30$\%$ more likely to defualt than someone with 4-8 credit cards.

\easysubproblem{Summarize Finlay's conception of \qu{big data}.}\spc{8}

According to Finlay there are 4 important features of 'Big Data' \begin{enumerate}
\item Volume-usually contains at least one terabyte of data
\item Variety-structured/unstrcutred, text, numbers..
\item Volatility-usually dynamic data, not static
\item Multi-sourced-data is gathered from all over.
\end{enumerate}

\end{enumerate}


\problem{This question is about probability estimation. We limit our discussion to estimating the probability that a single event occurs.}


\begin{enumerate}

\easysubproblem{What is the difference between the regression framework and the probability estimation framework?}\spc{6}

The outputs in regression are assigned real values, $\y\in\mathbb{R}$. Outputs in probability estimation are assigned probabilities, $\phat\in(0,1)$

\easysubproblem{Is probability estimation more similar to regression or classification and why?}\spc{3}

Probability estimation is more similar to classification than regression. Most likely, we will take the probability estimates calculated from the model to sort our data into classes: it will happen, it might happen, it wont happen.

\hardsubproblem{Why was it necessary to think of the response $Y$ as a random variable and why in particular the Bernoulli random variable?}\spc{3}

$Y$ is predetermined, it is either a 0 or 1. We can never be certain about what $Y$ truly is, we can only estimate it. Bernoulli random variable can be used as a binary classifier for $Y$, since it returns one of two values based on the $\phat$ input.

\hardsubproblem{If we use the Bernoulli r.v. for $Y$, are there any error terms (i.e. $\delta, \epsilon, e$) anymore? Yes/no.}\spc{0}

No. We can only estimate a random variable so the error is implied when we use $\approx$ instead of $=$.

\easysubproblem{What is the difference between $f$ in the regression framework and $f_{pr}$ in the probabilistic classification framework?}\spc{6}

$f$ in the regression framework assigns a real value to its response. $f_{pr}$ assigns the probability that its inputs will be a 1 .

\hardsubproblem{Is there a $t_{pr}$? If so, what does it look like?}\spc{3}

\begin{equation*}
t_{pr}(z_1,\cdots,z_t)=t(z_1,\cdots,z_t)
\end{equation*}

\easysubproblem{Write out the likelihood as a function of $f_{pr}$, the $\x_i$'s and the $y_i$'s.}\spc{3}
\begin{equation*}
P(Y_1, \cdots, Y_n)=\prod_{i=1}^nf_{pr}(\vec{x_i})(1-f_{pr}(\vec{x_i}))
\end{equation*}

\hardsubproblem{What assumption did you have to make and what would happen if you didn't make this assumption?}\spc{3}

We assumed that $Y_1,\cdots , Y_n$ were independent. If we didn't make this assumption, then we don't know the dependence structure of $Y_1,\cdots , Y_n$ and cannot use the $\Pi$ and neat formula.

\easysubproblem{Is $f_{pr}$ knowable? Yes/no.}\spc{0}

No, $f_{pr}$ is not knowable.
\end{enumerate}


\problem{This question continues the discussion of probability estimation for one event via the logistic regression approach.}


\begin{enumerate}

\intermediatesubproblem{As before, if we are to get anywhere at all, we need to approximate the true function $f_{pr}$ with a function in a hypothesis set, $\mathcal{H}_{pr}$. Let us examine the range of all elements in $\mathcal{H}_{pr}$. What values can these functions return and why?}\spc{4}

These functions can take on values between 0 and 1. The values can never be exactly 0 or 1 because we can never be 100$\%$ sure.

\hardsubproblem{We would also feel warm and fuzzy inside if the elements of $\mathcal{H}_{pr}$ contained the term $\w \cdot \x$.  What is the main reason we would like our prediction functions to contain this linear component?}\spc{1}

It's monotonically increasing and smooth, just like probabilities. 

\easysubproblem{The problem is $\w \cdot \x \in \reals$ but in (a) there is a special range of allowable functions. We need a way to transform $\w \cdot \x$ into the range from (a). What is this function called?}\spc{0}  

Link Function

\easysubproblem{Give some examples of such functions.}\spc{2}

\begin{eqnarray}
\phi(u)=\frac{e^u}{1+e^u}=\frac{1}{1+e^{-u}}\\
\phi(u)=1-e^{-e^u}\\
\phi(u)=\tanh(u)=\frac{e^u-e^{-u}}{e^u+e^{-u}}
\end{eqnarray}(1)Logistic (2)Complementary log-log (3) Hyperbolic tangent

\easysubproblem{We will choose the logistic function. Write the likelihood again from 2(g) but replace $f_{pr}$ with the element from $\mathcal{H}_{pr}$ that uses the logistic function.}\spc{2}

\begin{equation*}
P(Y_1,\cdots,Y_n)=\prod_{i=1}^n(\frac{\e^{\vec{w}\cdot\vec{x}}}{1+\e^{\vec{w}\cdot\vec{x}}})^{y_i}(1-\frac{\e^{\vec{w}\cdot\vec{x}}}{1+\e^{\vec{w}\cdot\vec{x}}})^{1-y_i}
\end{equation*}

\hardsubproblem{Simplify your answer from (e) so that you arrive at:
\beqn
\sum_{i=1}^n \natlog{1 + e^{(1 - 2y_i) \w \cdot \x_i}}
\eeqn}~\spc{10}
\begin{equation*}
argmax\{\prod_{i=1}^n(\frac{1}{1+\e^{-\w\cdot\x_i}})^{y_i}(\frac{1}{1+\e^{\w\cdot\x_i}})^{1-y_i}\}
\end{equation*}
Then if
\begin{eqnarray*}
y_i=1 \Rightarrow \prod_{i=1}^n(1+\e^{-\w\cdot\x_i})^{-1} &&
y_i=0 \Rightarrow \prod_{i=1}^n(1+\e^{\w\cdot\x_i})^{-1} \\
\end{eqnarray*}
\begin{equation*}
argmax\{\prod_{i=1}^n(1+\e^{(1-2y_i)\w\cdot\x_i})^{-1}\}=argmax\{-\sum\limits_{i=1}^n\natlog{1 + e^{(1 - 2y_i) \w \cdot \x_i}}
\end{equation*}
To get rid of the negative sign we can take the argmin instead of the argmax
\begin{equation*}
argmin\{\sum\limits_{i=1}^n\natlog{1 + e^{(1 - 2y_i) \w \cdot \x_i}}\}
\end{equation*}

\extracreditsubproblem{We will now maximize this likelihood w.r.t to $\w$ to find $\b$, the best fitting solution which will be used within $g_{pr}$ i.e.

\beqn
\b = \argmax_{\w \in \reals^{p+1}}\braces{\sum_{i=1}^n \natlog{1 + e^{(1 - 2y_i) \w \cdot \x_i}}}
\eeqn

to do so, we should find the derivative and set it equal to zero i.e.

\beqn
\derivop{\w}{\sum_{i=1}^n \natlog{1 + e^{(1 - 2y_i) \w \cdot \x_i}}} ~\buildrel \text{set} \over =~ 0
\eeqn

Try to find the derivate and solve. Get as far as you can. Do so on a separate page}\spc{0}

\easysubproblem{If you attempted the last problem, you found that there is no closed form solution. What type of methods are used to approximate $\b$? Note: once you use such methods and arrive at a $\b$, that is called \qu{running a logistic regression}. }\spc{1}

Numerical methods such as gradient descent.

\easysubproblem{In class we used the notation $\phat = g_{pr}$. Why?}\spc{1}

Since $g_{pr}$ is what we are using to approximate $f_{pr}$, it also returns a probability. 

\easysubproblem{Write down $\phat$ as a function of $\b$ and $\x$.}\spc{2}

\begin{equation*}
\phat=(1+e^{-\b\cdot\x})^{-1}
\end{equation*}

\intermediatesubproblem{What is the interpration of the linear component $\b \cdot \x$? What does it mean for $\phat$? No need to give the full, careful interpretation.}\spc{4}
\begin{equation*}
\b \cdot \x=\natlog{\frac{\phat}{1-\phat}}
\end{equation*}
It represents logodds$(\Y|\x)$

\hardsubproblem{How does one go about \textit{validating} a logistic regression model? What is the fundamental problem with doing so that you didn't have to face with regression or classification? Discuss.}\spc{4}

\end{enumerate}


\problem{This question is about probabilistic classification i.e. using probability estimation to classify. We limit our discussion to binary classification.}


\begin{enumerate}

\easysubproblem{How do you use a probability estimation model to classify. Provide the formula which provides $\hat{y}(\phat)$ i.e. the estimate of whether the event of interest occurs as a function of the probability estimate of the event occurring. Use the \qu{default} rule.}\spc{1}

  \begin{equation*}
  	\hat{y}=\mathbbm{1}_{\phat\leq0.5}
  \end{equation*}

\easysubproblem{In the formula from (a), there is an option to be made, write the formula again below with this option denoted $p_{th}$.}\spc{1}

  \begin{equation*}
      \hat{y}_i=\mathbbm{1}_{\phat_i\leq p_{th}}	
  \end{equation*}

\intermediatesubproblem{What happens when $p_{th}$ is low and what happens when $p_{th}$ is high? What is the tradeoff being made?}\spc{4}

The trade off will be between FP and FN. Low $p_{th}$ implies a high rate of False Positives and low rate of False Negatives. High $p_{th}$ implies high rate of False Negatives and low rate of False Positives.

\hardsubproblem{Below is the first 20 rows of in-sample prediction results from a logistic regression whose reponse is $>50K$ (the positive class) or $\leq 50K$ (the negative class). You have the $\phat_i$'s and the $y_i$'s. Create a performance table that includes the four numbers in the confusion table as well as FPR and recall. Leave some room for one additional column we will compute later in the question. The rows in the table should be indexed by $p_{th} \in \braces{0, 0.2, \ldots, 0.8, 1}$ which you should use as the first column. Hint: you may want to sort by $\phat$ and convert $y$ to binary before you begin.

\begin{table}[ht]
\footnotesize
\hspace{2cm}\begin{tabular}{ll}
  \hline
$\phat$ & $y$ \\ 
  \hline
0.35 & $>$50K \\ 
  0.49 & $>$50K \\ 
  0.73 & $>$50K \\ 
  0.91 & $>$50K \\ 
  0.01 & $<$=50K \\ 
  0.59 & $>$50K \\ 
  0.08 & $<$=50K \\ 
  0.07 & $<$=50K \\ 
  0.01 & $<$=50K \\ 
  0.76 & $>$50K \\ 
  0.32 & $<$=50K \\ 
  0.07 & $>$50K \\ 
  0.01 & $<$=50K \\ 
  0.00 & $<$=50K \\ 
  0.35 & $>$50K \\ 
  0.69 & $>$50K \\ 
  0.38 & $<$=50K \\ 
  0.07 & $<$=50K \\ 
  0.02 & $<$=50K \\ 
  0.00 & $<$=50K \\ 
   \hline
\end{tabular}
\end{table}
}~\spc{20}
\begin{figure}[htp]
\centering
\includegraphics[width=2in]{hw5t_45.jpeg}
\end{figure}

\intermediatesubproblem{Using the performance table from (d), trace out an approximate ROC curve.}\spc{6}

\begin{figure}[htp]
\centering
\includegraphics[width=2in]{hw5t_4e.jpeg}
\end{figure}

\intermediatesubproblem{Using the performance table from (d), trace out an approximate DET curve.}\spc{6}

\begin{figure}[htp]
\centering
\includegraphics[width=2in]{hw5t_4f.jpeg}
\end{figure}

\easysubproblem{Consider the $c_{FP} = \$5$ and $c_{FN} = \$1,000$. Explain how you would find the probabilistic classifier model that minimizes cost among the $p_{th}$ values you considered in your performance table in (d) but do not do any computations.}\spc{6}

Use the performance table to find the $p_{th}$ value with the lowest FN rate, that also has a reasonably low FP rate.

\end{enumerate}


\problem{These are questions related to bias-variance decomposition, bagging and random forests. }

\begin{enumerate}

\easysubproblem{List the assumptions for the bias-variance decomposition.}\spc{3}
\setcounter{equation}{0}
\begin{eqnarray}
E[Y|X=x]=f(x)\\
Var[\Delta|X=x]=Var[\Delta]=\sigma^2
\end{eqnarray}

\intermediatesubproblem{Why is $f(\x)$ called the \qu{conditional expectation function}?}\spc{3}

It approximates the expected value of the function $Y$ given $x$

\easysubproblem{Provide an expression for the bias-variance decomposition formula for the average MSE over the distribution $\prob{\X}$ for $y = g + (f-g) + \delta$. You should have three terms in the expression. Make sure you explain conceptually each term in English.}\spc{3}

\begin{equation*}
	MSE=\sigma^2+E_\mathcal{X}[Var[g(\vec{x})]]+E_\mathcal{X}[Bias[g(\vec{x})]^2]	
\end{equation*}
$\sigma^2$ is the irreducible error. $E_\mathcal{X}[Bias[g(\vec{x})]^2]$ is how far off the $g$ is from $f$, on average. $E_\mathcal{X}[Var[g(\vec{x})]]$ is how much
\extracreditsubproblem{Rederive the bias-variance decomposition formula for the average MSE over the distribution $\prob{\X}$ for $y = g + (h^* - g) + (f - h^*) + \delta$. You should group the final expression into \emph{four} terms where two will be the same as the expression found in (c), one will be similar to a term found in (c) and one will be new. Make sure you explain conceptually each term in English. Do so on an additional page.}\spc{-0.5}

\intermediatesubproblem{Assume a $\mathbb{D}$ where $n$ is large and $p$ is small and you fit a linear model $g$ to all features. Your in-sample $R^2$ is low. In the expression from (c), indicate term(s) are likely large, which term(s) are likely small and explain why.}\spc{6}

Variance will be small due to large $n$ and the rest of the reducible error will be in the Bias term.
\intermediatesubproblem{Assume a $\mathbb{D}$ where $n$ is large and $p$ is small and you fit a tree model $g$ to all features. Your in-sample $R^2$ is low. In the expression from (c), indicate term(s) are likely large, which term(s) are likely small and explain why.}\spc{6}

Trees keep bias low but have very high variance from tree to tree. 

\easysubproblem{Provide an expression for the bias-variance decomposition formula for the average MSE over the distribution $\prob{\X}$ for $y = g + (f-g) + \delta$ where $g$ now represents the average taken over constituent models $g_1, g_2, \ldots, g_T$. (This is known as \qu{model averaging} or \qu{ensemble learning}). You can assume that $\rho := \corr{g_{t_1}}{g_{t_2}}$ is the same for all $t_1 \neq t_2$.}\spc{4}

\begin{equation*}
MSE=\sigma^2+\rho Var[g_t]+\frac{1-\rho}{T}Var[g_t]+Bias[g_t]^2
\end{equation*}
\easysubproblem{If $T \rightarrow \infty$, rewrite the bias-variance decomposition you found in (k).}\spc{2}
\begin{equation*}
\sigma^2+\rho Var[g_t]+Bias[g_t]^2
\end{equation*}

\easysubproblem{If $g_1, g_2, \ldots, g_T$ are built with the same data $\mathbb{D}$ and $\mathcal{A}$ is not random, then $g_1 = g_2 = \ldots = g_T$. What would $\rho$ be in this case?}\spc{3}
\begin{equation*}
\rho=1
\end{equation*}

\easysubproblem{Even though each of the constituent models $g_1, g_2, \ldots, g_T$ are built with the same data $\mathbb{D}$, what idea can you use to induce $\rho < 1$? This idea is called \qu{bagging} which is a whimsical portmanteau of the words \qu{bootstrap aggregation}.}\spc{4}

We can build each $\mathbb{D}_{train_t}, \mathbb{D}_{test_t}$ by sampling $\mathbb{D}$ with replacement. $g_{bag}$ is then the average of all models $g_1,\cdots,g_T$.

\easysubproblem{Explain how examining predictions averaged on the out of bag (oob) data for each $g_1, g_2, \ldots, g_T$ can constitute model validation for the bagged model.}\spc{4}

We can validate each model individually by testing it on the data that was left "out of the bag" when we were training the model. We can average all of these 'oob' error statistics together to get $oob_{bag}$ and use this to validate $g_{bag}$

\easysubproblem{Explain how the Random Forests$^\circledR$  algorithm differs from the CART (classification and regression trees) algorithm.}\spc{4}

The CART algorithm uses all features/splits, Random Forests use a randomized subset of features.?
\easysubproblem{Explain why the MSE for the Random Forests$^\circledR$ algorithm expected to be better than a bag of CART models.}\spc{4}

It's more random.

\easysubproblem{List the three major advantages of Random Forests$^\circledR$ for supervised learning / machine learning.}\spc{4}
You get something for nothing

\end{enumerate}

\problem{These are questions related to correlation, causation and the interpretation of coefficients in linear models / logistic regression. }

\begin{enumerate}

\easysubproblem{You are provided with the responses measured from a phenomenon of interest $y_1, \ldots, y_n$ and associated measurements $x_1, \ldots, x_n$ where $n$ is large. The sample correlation is estimated to be $r=0.74$. Is $\x$ \qu{correlated} with $\y$?}\spc{1}

Yes?

\intermediatesubproblem{Consider the case in (a), would $\x$ be a \qu{causal} factor for $\y$? Explain.}\spc{3}

It's impossible to tell.
\intermediatesubproblem{Consider the case in (a) and create two plausible causal models using the graphical depiction style used in class (nodes representing variables and lines represent causal contribution where node A below node B means node A is measured before node B). Your model has to include $x$ and $y$ but is not limited to only those variables.}\spc{8}

\begin{figure}[htp]
\centering
\includegraphics[width=2in]{hw5t_6c.jpeg}
\end{figure}

\intermediatesubproblem{Consider the case in (a) but now $n$ is small. Create a third plausible causal model (in addition to the two you created in the last problem) using the same graphical depiction style. Your model has to include $x$ and $y$ but is not limited to only those variables.}\spc{5}

\begin{figure}[htp]
\centering
\includegraphics[width=2in]{hw5t_6d.jpeg}
\end{figure}

\easysubproblem{Explain briefly how you would prove beyond a reasonable doubt that $\x$ is not only correlated with $\y$ but that $\x$ is a causal factor of $\y$.}\spc{3}

We can keep everything constant in the system, manipulate $x$ and if we see a chance in $\y$, then we can say $x$ is a causal factor of $\y$

\easysubproblem{Consider $\x$ is college GPA and $\y$ is career average income. Is $\x$ correlated with $\y$? Do not lookup data online, I want you to answer conceptually using your own argument.}\spc{3}

$x$ is slightly correlated with GPA. Most universities have a minimum threshold GPA score students must maintain in order to graduate. If a students GPA is below this threshold, even though the score is $>0$, the student might not be allowed to graduate. A person with a GPA of 1.0 indicates more laziness than a person GPA of 0.0. There could be a person who never went to college, so their GPA=0, but they have a learned trade(plumbing). The plumber probably makes more money than the college dropout.  


\intermediatesubproblem{Consider $\x$ is college GPA and $\y$ is career average income. Is $\x$ a causal factor of $\y$? Do not lookup data online, I want you to answer conceptually using your own argument.}\spc{3}

No,if we kept everything constant in the system and increaed a persons GPA by a certain amount, their average income would remain the same.

\intermediatesubproblem{Consider $\x$ is college GPA and $\y$ is career average income. Can you think of a $\z$ which is a lurking variable? Explain the variable and why you believe it fits the description of a lurking variable.}\spc{3}

Personality type. Having a high GPA is not about IQ, you must care about certain things and follow certain rules(handing in assignments on time, attendance). The drive behind a persons desire to achieve and MAINTAINE a high GPA won't dissapear after they graduate college. They will likely still feel the need to perform well in the eyes of others, and take an office job somewhere to wither and die.

\intermediatesubproblem{If you fit a linear model for $\y$, $g = b_0 + b_x x + b_z z$, what would the $b_x$ value be close to? Why?}\spc{3}

\extracreditsubproblem{Create a causal model using the same graphical depiction style that justifies the four linear regression assumptions. Do so on a different page.}\spc{-0.6}


\intermediatesubproblem{When running a regression of \texttt{price} on all variables in the \texttt{diamonds} dataset, the coefficient for \texttt{carat} is about \$6,500. Interpret this value as best as you can.}\spc{4}

For two diamonds A and B that were observed in the same way as the diamonds dataset, if A has a value of carat one unit larger than the carat value of B, then the price of diamond A is predicted to be \$6500 more than the price of diamond B.

\intermediatesubproblem{When running a logistic regression of class \texttt{malignant} on all variables in the \texttt{biopsy} dataset, the coefficient for \texttt{V1} (which measures clump thickness) is about 0.54. Interpret this value as best as you can.}\spc{5}

For two observations A and B that were sampled in the same way as the data in the biopsy data set, if A's clump thickness is greater than Bs by a value equal to V1, then A is .54 times more likely to be malignant than B,

\end{enumerate}

\end{document}








































\problem{These are questions about Silver's book, chapters ...  For all parts in this question, answer using notation from class (i.e. $t ,f, g, h^*, \delta, \epsilon, e, t, z_1, \ldots, z_t, \mathbb{D}, \mathcal{H}, \mathcal{A}, \mathcal{X}, \mathcal{Y}, X, y, n, p, x_{\cdot 1}, \ldots, x_{\cdot p}$, $x_{1 \cdot}, \ldots, x_{n \cdot}$, etc. and also we now have $f_{pr}, h^*_{pr}, g_{pr}, p_{th}$, etc from probabilistic classification as well as different types of validation schemes). }

\begin{enumerate}

\easysubproblem{What algorithm that we studied in class is PECOTA most similar to?}\spc{1}

\easysubproblem{Is baseball performance as a function of age a linear model? Discuss.}\spc{2}

\intermediatesubproblem{How can baseball scouts do better than a prediction system like PECOTA?}\spc{4}

\intermediatesubproblem{Why hasn't anyone (at the time of the writing of Silver's book) taken advantage of Pitch f/x data to predict future success?}\spc{4}

\hardsubproblem{Chapter 4 is all about predicting weather. Broadly speaking, what is the problem with weather predictions? Make sure you use the framework and notation from class. This is not an easy question and we will discuss in class. Do your best.}\spc{6}

\easysubproblem{Why does the weatherman lie about the chance of rain? And where should you go if you want honest forecasts?}\spc{2}

\hardsubproblem{Chapter 5 is all about predicting earthquakes. Broadly speaking, what is the problem with earthquake predictions? It is \textit{not} the same as the problem of predicting weather. Read page 162 a few times. Make sure you use the framework and notation from class.}\spc{6}

\easysubproblem{Silver has quite a whimsical explanation of overfitting on page 163 but it is really educational! What is the nonsense predictor in the model he describes?}\spc{2}


\easysubproblem{John von Neumann was credited with saying that \qu{with four parameters I can fit an elephant and with five I can make him wiggle his trunk}. What did he mean by that and what is the message to you, the budding data scientist? }\spc{5}

\hardsubproblem{Chapter 6 is all about predicting unemployment, an index of macroeconomic performance of a country. Broadly speaking, what is the problem with unemployment predictions? It is \textit{not} the same as the problem of predicting weather or earthquakes. Make sure you use the framework and notation from class.}\spc{6}

\extracreditsubproblem{Many times in this chapter Silver says something on the order of \qu{you need to have theories about how things function in order to make good predictions.} Do you agree? Discuss.}\spc{4}


\end{enumerate}


\problem{This question is about validation for the supervised learning problem with one fixed $\mathbb{D}$.}


\begin{enumerate}

\easysubproblem{For one fixed $\mathcal{H}$ and $\mathcal{A}$ (i.e. one model), write below the steps to do a simple validation and include the final step which is shipping the final $g$.}\spc{6}

\easysubproblem{For one fixed $\mathcal{H}$ and $\mathcal{A}$ (i.e. one model), write below the steps to do a $K$-fold cross validation and include the final step which is shipping the final $g$.}\spc{10}

\intermediatesubproblem{For one fixed $\mathcal{H}$ and $\mathcal{A}$ (i.e. one model), write below the steps to do a bootstrap validation and include the final step which is shipping the final $g$.}\spc{10}

\intermediatesubproblem{For one fixed $\mathcal{H}_1, \ldots \mathcal{H}_M$ and $\mathcal{A}$ (i.e. $M$ different models), write below the steps to do a simple validation and include the final step which is shipping the final $g$.}\spc{22}

\hardsubproblem{For one fixed $\mathcal{H}_1, \ldots \mathcal{H}_M$ and $\mathcal{A}$ (i.e. $M$ different models), write below the steps to do a $K$-fold cross validation and include the final step which is shipping the final $g$. This is not an easy problem! There are a lot of steps and a lot to keep track of...}\spc{22}


\end{enumerate}


\problem{This question is about ridge regression --- an alternative to OLS.}


\begin{enumerate}

\intermediatesubproblem{Imagine we are in the \qu{Luis situation} where we have $\X$ with dimension $n \times (p+1)$ but $p+1 > n$ and we still want to do OLS. Why would the OLS solution we found previously break down in this case?}\spc{4}

\intermediatesubproblem{We will embark now to provide a solution for this case. The solution will also give nice results for other situations besides the Luis situation as well. First, assume $\lambda$ is a positive constant and demonstrate that the expression $\lambda \normsq{\w} = \w^\top (\lambda \I) \w$ i.e. it can be expressed as a quadratic form where $\lambda\I$ is the determining matrix. We will call this term $\lambda \normsq{\w}$ the \qu{ridge penalty}.}\spc{3}


\easysubproblem{Write the $\mathcal{H}$ for OLS below where there parameter is the $\w$ vector. $\w \in$ ?}\spc{1}

\easysubproblem{Write the error objective function that OLS minimizes using vectors, then expand the terms similar to the previous homework assignment.}\spc{1}

\easysubproblem{Now add the ridge penalty $\lambda \normsq{\w}$ to the expanded form you just found and write it below. We will term this two-part error function the \qu{ridge objective}.}\spc{1}

\easysubproblem{Note that the ridge objective looks a bit like the hinge loss we spoke about when we were learning about support vector machines. There are two pieces of this error function in counterbalance. When this is minimized, describe conceptually what is going on.}\spc{5}

\intermediatesubproblem{Now, the ridge penalty term as a quadratic form can be combined with the last term in the least squares error from OLS. Do this, then use the rules of vector derivatives we learned to take $d/d\w$ and write the answer below.}\spc{2}

\easysubproblem{Now set that derivative equal to zero. What matrix needs to be invertible to solve?}\spc{2}

\hardsubproblem{There's a theorem that says \textit{positive definite} matrices are invertible. A matrix is said to be positive definite if every quadratic form is positive for all vectors i.e. if $\forall \z \neq \zerovec~~ \z^\top A \z > 0$ then $A$ is positive definite. Prove this matrix from the previous question is positive definite.}\spc{5}

\easysubproblem{Now that it's positive definite (and thus invertible), solve for the $\w$ that is the argmin of the ridge objective, call it $\b_{ridge}$. Note that this is called the \qu{ridge estimator} and computing it is called \qu{ridge regression} and it was invented by Hoerl and Kennard in 1970.}\spc{3}


\easysubproblem{Did we just figure out a way out of Luis's situation? Explain.}\spc{3}

\intermediatesubproblem{It turns out in the Luis situation, many of the values of the entries of $\b_{ridge}$ are close to 0. Why should that be? Can you explain now conceptually how ridge regression works?}\spc{3}

\easysubproblem{Find $\yhat$ as a function of $\y$ using $\b_{ridge}$. Is $\yhat$ an orthogonal projection of $\y$ onto the column space of $\X$?}\spc{3}

\extracreditsubproblem{Show that this $\yhat$ is an orthogonal projection of $\y$ onto the column space of some matrix $\X_{ridge}$ (which is not $\X$!) and explain how to construct $\X_{ridge}$ on a separate page.}\spc{0}

\easysubproblem{Is the $\mathcal{H}$ for OLS the same as the $\mathcal{H}$ for ridge regression? Yes/no. \\ Is the $\mathcal{A}$ for OLS the same as the $\mathcal{A}$ for ridge regression? Yes/no.}\spc{-0.5}

\intermediatesubproblem{What is a good way to pick the value of $\lambda$, the hyperparameter of the $\mathcal{A}$ = ridge?}\spc{1}




\easysubproblem{In classification via $\mathcal{A}$ = support vector machines with hinge loss, how should we pick the value of $\lambda$? Hint: same as previous question!}\spc{1}



\extracreditsubproblem{Besides the Luis situation, in what other situations will ridge regression save the day?}\spc{3}


\hardsubproblem{The ridge penalty is beautiful because you were able to take the derivative and get an analytical solution. Consider the following algorithm:

\beqn
\b_{lasso} = \argmin_{\w~\in~\reals^{p + 1}} \braces{(\y - \X\w)^\top(\y - \X\w) + \lambda \norm{\w}^1}
\eeqn

This penalty is called the \qu{lasso penalty} and it is different from the ridge penalty in that it is not the norm of $\w$ squared but just the norm of $\w$. It turns out this algorithm (even though it has no closed form analytic solution and must be solved numerically a la the SVM) is very useful! In \qu{lasso regression} the values of $\b_{lasso}$ are not shrunk \textit{towards} 0 they are harshly punished \textit{directly to} 0! How do you think lasso regression would be useful in data science? Feel free to look at the Internet and write a few sentences below.}~\spc{6}

\easysubproblem{Is the $\mathcal{H}$ for OLS the same as the $\mathcal{H}$ for lasso regression? Yes/no. \\ Is the $\mathcal{A}$ for OLS the same as the $\mathcal{A}$ for lasso regression? Yes/no.}\spc{-0.5}

\end{enumerate}

\problem{These are questions about non-parametric regression.}


\begin{enumerate}

\easysubproblem{In problem 1, we talked about schemes to validate algorithms which tried $M$ different prespecified models. Where did these models come from?}\spc{4}

\intermediatesubproblem{What is the weakness in using $M$ pre-specified models?}\spc{5}

\hardsubproblem{Explain the steps clearly in forward stepwise linear regression.}\spc{6}

\hardsubproblem{Explain the steps clearly in \emph{backwards} stepwise linear regression.}\spc{7}

\intermediatesubproblem{What is the weakness(es) in this stepwise procedure?}\spc{4}

\easysubproblem{Define \qu{non-parametric regression}. What problem(s) does it solve? What are its goals? Discuss.}\spc{7}

\intermediatesubproblem{Provide the steps for the regression tree (the one algorithm we discussed in class) below.}\spc{10}


\easysubproblem{Consider the following data 

\begin{figure}[htp]
\centering
\includegraphics[width=3in]{curvy}
\end{figure}

Create a tree with maximum depth 1 (i.e one split at the root node) and plot $g$ above.}~\spc{4}


\easysubproblem{Now add a second split to the tree and plot $g$ below.

\begin{figure}[htp]
\centering
\includegraphics[width=3in]{curvy}
\end{figure}

}~\spc{-0.5}


\easysubproblem{Now add a third split to the tree and plot $g$ below.

\begin{figure}[htp]
\centering
\includegraphics[width=3in]{curvy}
\end{figure}

}~\spc{5}


\easysubproblem{Now add a fourth split to the tree and plot $g$ below.

\begin{figure}[htp]
\centering
\includegraphics[width=3in]{curvy}
\end{figure}

}~\spc{-0.5}

\easysubproblem{Draw a tree diagram of $g$ below indicating which nodes are the root, inner nodes and leaves. Indicate split rules and leaf values clearly.}\spc{15}



\easysubproblem{Plot $g$ below for the mature tree with the default $N_0 =$ \texttt{nodesize} hyperparameter.

\begin{figure}[htp]
\centering
\includegraphics[width=3in]{curvy}
\end{figure}

}~\spc{-0.5}


\easysubproblem{If $N_0 =1$, what would likely go wrong?}\spc{2}

\easysubproblem{How should you pick the $N_0 =$ \texttt{nodesize} hyperparameter in practice?}\spc{2}


\end{enumerate}


\problem{These are questions about classification trees.}


\begin{enumerate}

\easysubproblem{How are classification trees different than regression trees?}\spc{2}

\intermediatesubproblem{What are the steps in the classification tree algorithm?}\spc{12}


\end{enumerate}

\problem{These are questions about measuring performance of a classifier.}

\begin{enumerate}

\easysubproblem{What is a confusion table?}\spc{8}


Consider the following in-sample confusion table where \qu{$>50$K} is the positive class:

\begin{Verbatim}
       y_hats_train
y_train <=50K >50K
  <=50K  3475  262
  >50K    471  792
\end{Verbatim}

\easysubproblem{Calculate the following: $n$ (sample size) = \\~\\
$FP$ (false positives) = \\~\\
$TP$ (true positives) = \\~\\
$FN$ (false negatives) = \\~\\
$TN$ (true negatives) = \\~\\
$\#P$ (number positive) = \\~\\
$\#N$ (number negative) = \\~\\
$\#PP$ (number predicted positive) = \\~\\
$\#PN$ (number predicted negative) = \\~\\
$\#P / n$ (prevalence / marginal rate / base rate) = \\~\\
$(FP + FN) / n$ (misclassification error) = \\~\\
$(TP + TN) / n$ (accuracy) = \\~\\
$TP / \#PP$ (precision) = \\~\\
$TP / \#P$ (recall, sensitivity, true positive rate, TPR) = \\~\\
$2 / (\text{recall}^{-1} + \text{precision}^{-1})$ (F1 score) = \\~\\
$FP / \#PP$ (false discovery rate, FDR) = \\~\\
$FP / \#N$ (false positive rate, FPR) = \\~\\ %false alarm rate 
$FN / \#PN$ (false omission rate, FOR) = \\~\\
$FN / \#P$ (false negative rate, FNR) = %miss rate 
}

\easysubproblem{Why is FPR also called the \qu{false alarm rate}?}\spc{4}

\easysubproblem{Why is FNR also called the \qu{miss rate}?}\spc{4}


\easysubproblem{Below let the red icons be the positive class and the blue icons be the negative class. 


\begin{figure}[htp]
\centering
\includegraphics[width=1.5in]{precision_recall.jpg}
\end{figure}

The icons included inside the black border are those that have $\hat{y} = 1$. Compute both precision and recall.}\spc{4}

\intermediatesubproblem{There is always a tradeoff of FP vs FN. However, in some situations, you will look at FPR vs. FNR. Describe such a classification scenario. It does not have to be this income amount classification problem, it can be any problem you can think of.}\spc{3}

\intermediatesubproblem{There is always a tradeoff of FP vs FN. However, in some situations, you will look at FDR vs. FOR. Describe such a classification scenario. It does not have to be this income amount classification problem, it can be any problem you can think of.}\spc{3}

\intermediatesubproblem{There is always a tradeoff of FP vs FN. However, in some situations, you will look at precision vs. recall. Describe such a classification scenario. It does not have to be this income amount classification problem, it can be any problem you can think of.}\spc{3}


\intermediatesubproblem{There is always a tradeoff of FP vs FN. However, in some situations, you will look only at an overall metric such as accuracy (or $F1$). Describe such a classification scenario. It does not have to be this income amount classification problem, it can be any problem you can think of.}\spc{4}

\end{enumerate}







\end{document}

\problem{These are questions about Silver's book, chapter 2.}


\begin{enumerate}

\intermediatesubproblem{If one's goal is to fit a model for a phenomenon $y$, what is the difference between the approaches of the hedgehog and the fox? Answer using notation from class (i.e. $t ,f, g, h^*, \delta, \epsilon, e, t, z_1, \ldots, z_t, \mathbb{D}, \mathcal{H}, \mathcal{A}, \mathcal{X}, \mathcal{Y}, X, y, n, p, x_{\cdot 1}, \ldots, x_{\cdot p}, x_{1 \cdot}, \ldots, x_{n \cdot}$, etc.). Connecting this to the modeling framework should really make you think about what Tetlock's observation means for political and historical phenomena.}\spc{4}

\easysubproblem{Why did Harry Truman like hedgehogs? Are there a lot of people that think this way?}\spc{4}


\hardsubproblem{Why is it that the more education one acquires, the less accurate one's predictions become?}\spc{4}


\easysubproblem{Why are probabilistic classifiers (i.e. algorithms that output functions that return probabilities) better than vanilla classifiers (i.e. algorithms that only return the class label)? We will move in this direction in class soon.}\spc{4}

\end{enumerate}

\problem{These are questions about Finlay's book, chapter 2-4. We will hold off on chapter 1 until we cover probability estimation after midterm 2.}


\begin{enumerate}

\easysubproblem{What term did we use in class for \qu{behavioral (outome) data}?}\spc{0}

\easysubproblem{Write about some reasons why data scientists implement models that are subpar in predictive performance (p27).}\spc{3}


\easysubproblem{In the first wine example, what is the outcome metric and what kind of supervised learning was employed?}\spc{0}

\easysubproblem{In the second wine example, what is the outcome metric and kind of supervised learning was employed?}\spc{0}


\easysubproblem{In the third chapter, why is it that some organizations cannot use predictive modeling to improve their business?}\spc{3}

\easysubproblem{In the bankruptcy case, what is the problem with merely using $g$ to obtain a $\hat{y}$ without any other information from the model?}\spc{3}

\easysubproblem{Chapter 3 talks about using the model with human judgment. Under what circumstances is this beneficial? Answer using notation from class (i.e. $t ,f, g, h^*, \delta, \epsilon, e, t$, $z_1, \ldots, z_t, \mathbb{D}, \mathcal{H}, \mathcal{A}, \mathcal{X}, \mathcal{Y}, X, y, n, p, x_{\cdot 1}, \ldots, x_{\cdot p}, x_{1 \cdot}, \ldots, x_{n \cdot}$, etc.).}\spc{3}


\hardsubproblem{In Chapter 4 Finaly makes an interesting observation based on his experience in data science. He says most predictive models have $p \leq 30$. Why do you think this is? Discuss.}\spc{5}


\easysubproblem{He says there is \qu{almost always other data that could be acquired ... [which] doesn't always come for free}. The \qu{data} he is talking about here specifically means \qu{more predictors} i.e. increasing $p$. In what cases would someone be willing to pay for this data?}\spc{3}


\easysubproblem{Table 4 lists \qu{data types} about what type of observations?}\spc{1}

\easysubproblem{What type of data does he find in his experience to be the most important to predictive modeling? Why do you think this is so?}\spc{3}

\easysubproblem{If $x_{\cdot 17}$ was age and $x_{\cdot 18}$ is age of spouse, what is the most likely reason why adding $x_{\cdot 18}$ to $\mathbb{D}$ not be friutful for predictive ability?}\spc{3}

\hardsubproblem{What is the lifespan of a predictive model? Why does it not last forever? Answer using notation from class (i.e. $t ,f, g, h^*, \delta, \epsilon, e, t$, $z_1, \ldots, z_t, \mathbb{D}, \mathcal{H}, \mathcal{A}, \mathcal{X}, \mathcal{Y}, X, y, n, p$, $x_{\cdot 1}, \ldots, x_{\cdot p}, x_{1 \cdot}, \ldots, x_{n \cdot}$, etc.).}\spc{3}


\hardsubproblem{What does \qu{large enough to representative of the full population} (p80) mean? Answer using notation from class (i.e. $t ,f, g, h^*, \delta, \epsilon, e, t$, $z_1, \ldots, z_t, \mathbb{D}, \mathcal{H}, \mathcal{A}, \mathcal{X}, \mathcal{Y}, X, y, n, p$, $x_{\cdot 1}, \ldots, x_{\cdot p}, x_{1 \cdot}, \ldots, x_{n \cdot}$, etc.).}\spc{3}

\easysubproblem{Is there a hype about \qu{big data} i.e. including millions of observations instead of a few thousand? Discuss Finlay's opinion.}\spc{3}


\easysubproblem{What is Finlay's solution to \qu{overfitting} (p84)?}\spc{5}
\end{enumerate}


\problem{These are questions about association and correlation.}


\begin{enumerate}

\easysubproblem{Give an example of two variables that are both correlated and associated by drawing a plot.}\spc{4}

\easysubproblem{Give an example of two variables that are not correlated but are associated by drawing a plot.}\spc{4}

\easysubproblem{Give an example of two variables that are not correlated nor associated by drawing a plot.}\spc{4}

\easysubproblem{Can two variables be correlated but not associated? Explain.}\spc{4}


\end{enumerate}

\problem{These are questions about multivariate linear model fitting using the least squares algorithm.}

\begin{enumerate}

\hardsubproblem{Derive $\partialop{\c}{\c^\top A \c}$ where $\c \in \reals^n$ and $A \in \reals^{n \times n}$ but \textit{not} symmetric. Get as far as you can.}\spc{8}

\easysubproblem{Given matrix $X \in \reals^{n \times (p+1)}$, full rank and first column consisting of the $\onevec_n$ vector, rederive the least squares solution $\b$ (the vector of coefficients in the linear model shipped in the prediction function $g$). No need to rederive the facts about vector derivatives.}\spc{10}

\intermediatesubproblem{Consider the case where $p = 1$. Show that the solution for $\b$ you just derived is the same solution that we proved for simple regression in Lecture 8. That is, the first element of $\b$ is the same as $b_0 = \ybar - r \frac{s_y}{s_x}\xbar$ and the second element of $\b$ is $b_1 = r \frac{s_y}{s_x}$.} \spc{10}

\easysubproblem{If $X$ is rank deficient, how can you solve for $\b$? Explain in English.} \spc{2}

\hardsubproblem{Prove $\rank{X} =\rank{X^\top X}$.}\spc{6}

\hardsubproblem{Given matrix $X \in \reals^{n \times (p+1)}$, full rank and first column consisting of the $\onevec_n$ vector, now consider cost multiples (\qu{weights}) $c_1, c_2, \ldots, c_n$ for each mistake $e_i$. As an example, previously the mistake for the 17th observation was $e_{17} := y_{17} - \hat{y}_{17}$ but now it would be $e_{17} := c_{17} (y_{17} - \hat{y}_{17})$.  Derive the weighted least squares solution $\b$. No need to rederive the facts about vector derivatives. Hints: (1) show that SSE is a quadratic form with the matrix $C$ in the middle (2) Split this matrix up into two pieces i.e. $C = C^{\half} C^{\half}$, distribute and then foil (3) note that a scalar value equals its own transpose and (4) use the vector derivative formulas.}\spc{20}


\hardsubproblem{If $p=1$, prove $r^2 = R^2$ i.e. the linear correlation is the same as proportion of sample variance explained in a least squares linear model.}\spc{6}


\intermediatesubproblem{Prove that the point $<1,\xbar_1, \xbar_2, \ldots, \xbar_p, \bar{y}>$ is a point on the least squares linear solution.}\spc{13}

\end{enumerate}

\problem{These are questions related to the concept of orthogonal projection, QR decomposition and its relationship with least squares linear modeling.}

\begin{enumerate}

\easysubproblem{Consider least squares linear regression using a design matrix $X$ with rank $p + 1$. What are the degrees of freedom in the resulting model? What does this mean?}\spc{3}


\intermediatesubproblem{If you are orthogonally projecting the vector $\y$ onto the column space of $X$ which is of rank $p + 1$, derive the formula for $\proj{\colsp{X}}{\y}$. Is this the same as the least squares solution?}\spc{6}

\hardsubproblem{We saw that the perceptron is an \textit{iterative algorithm}. This means that it goes through multiple iterations in order to converge to a closer and closer $\w$. Why not do the same with linear least squares regression? Consider the following. Regress $\y$ using $\X$ to get $\yhat$. This generates residuals $\e$ (the leftover piece of $\y$ that wasn't explained by the regression's fit, $\yhat$). Now try again! Regress $\e$ using $\X$ and then get new residuals $\e_{new}$. Would $\e_{new}$ be closer to $\zerovec_n$ than the first $\e$? That is, wouldn't this yield a better model on iteration \#2? Yes/no and explain.}\spc{10}


\intermediatesubproblem{Prove that $Q^\top = Q^{-1}$ where $Q$ is an orthonormal matrix such that $\colsp{Q} = \colsp{X}$ and $Q$ and $X$ are both matrices $\in \reals^{n \times (p+1)}$. Hint: this is purely a linear algebra exercise.}\spc{10}


\intermediatesubproblem{Prove that the least squares projection $H = \XXtXinvXt$ is the same as $QQ^\top$.}\spc{10}

\intermediatesubproblem{Prove that an orthogonal projection onto the $\colsp{Q}$ is the same as the sum of the projections onto each column of $Q$.}\spc{10}


\hardsubproblem{Trouble in paradise. Prove that the SSE of a multivariate linear least squares model always decreases (equivalently, $R^2$ always increases) upon the addition of a new independent predictor. Keep in mind this holds true even if this new predictor has no information about the true causal inputs to the phenomenon $y$.}\spc{12}

\intermediatesubproblem{Why is this a bad thing? Explain in English.}\spc{3}



\extracreditsubproblem{Prove that $\rank{H} =\tr{H}$.}\spc{-0.5}

\end{enumerate}


\end{document}